\subsection*{Clarinete}

A clarineta, ou clarinete, faz parte da família dos instrumentos de
madeira e seu som é produzido através da vibração de uma palheta
simples. Foi inventada por Johann Denner por volta de 1700, quando
transformou um chalumeau de palhetas duplas em um instrumento de
palheta simples. A palavra clarineta vem do italiano clarino, um
antigo tipo de trompete agudo. Em 1839, dois fabricantes de clarineta,
Klosé e Buffet, criaram uma clarineta segundo o mecanismo de chaves
inventado para flauta por Theobald Boehm na primeira metade do século
XIX. A família da clarineta inclui as clarinetas soprano (em si bemol,
que é a mais comum, lá e dó), a requinta em mi bemol (a mais aguda), a
clarineta alto em mi bemol, o clarone (ou clarineta baixo) em si
bemol, a clarineta contralto em mi bemol e a clarineta contrabaixo em
si bemol. Ela é um instrumento muito versátil, fazendo parte dos mais
variados grupos instrumentais da música erudita e popular (orquestra
sinfônica, banda e grupos de choro, jazz e música folclórica).