\tiny{.} \normalsize

Viva o ensino da Música! É com muita felicidade que após mais
de uma década de trabalho como conselheiro da ABEMÚSICA (Associação
Brasileira de Música) conseguidos a aprovação da Lei que garante a
obrigatoriedade do ensino da Música em todo o país, tendo como limite
de implantação o ano de 2012.\\


Foi sem dúvida um trabalho árduo que no decorrer do processo recebeu
críticas, sugestões e ajuda de muitas outras entidades.\\


Tornando obrigatório o Ensino da Música nos deparamos com algumas
perguntas: Quem dará as aulas de Música? O Estado comprará os
instrumentos? O que ensinar? Que método utilizar? Ou seja, faz parte
deste processo estas indagações e adequações do ensino da Música no
Brasil, valorizando as qualidades rítmicas, de timbres, de estilos que
cada região brasileira tem.\\


Num primeiro momento apareceram aqueles que gostariam de ressuscitar o
Canto Orfeônico, outros obrigar o estudo apenas da música erudita,
outros afirmando que ouvir uma música no CD já seria uma aula de
música, ou seja, ninguém se entendia e o governo não se pronunciava,
ou melhor, deixou muito a vontade, ao diretor da escola a decisão de
como fazer música em sua unidade de ensino. Temos a obrigação moral de
estar acompanhando estes movimentos perto de nós, observando se estão
realmente ensinando música aos nossos jovens.\\


Diante de tantas mudanças, nossa editora, que sempre esteve
diretamente ligada ao ensino musical, com seus inúmeros métodos de
ensino, dá um passo a frente tornando realidade esta magnífica coleção
\underline{DA CAPO CRIATIVIDADE}.\\


Em 2004 lançamos com muito sucesso o DA CAPO, utilizado por todo o
Brasil como o método mais dinâmico do Ensino Coletivo de Instrumentos
de Banda. Agora, nesta nova coleção, o Maestro Joel Barbosa ensina aos
nossos jovens músicos a improvisação dentro da coletividade de uma
Banda.\\


Este material é único e vem em ótima hora, visto que os instrumentos
de Banda muitas vezes já existem nas escolas, jogados em salas de
almoxarifado e sem utilização.\\


Está na hora, caros estudantes e professores, de buscarmos o estudo da
Música com Instrumentos de Banda, recuperando nosso instrumentos,
estudando com um método eficaz, \underline{DA CAPO CRIATIVIDADE},
transformando a
vida de jovens pelo Brasil.\\



\begin{flushleft}
  Maestro Marcelo Fagundes\\
  Editor\\
  Keyboard Editora Musical Ltda.\\
\end{flushleft}
\break
\tiny{.} \normalsize
