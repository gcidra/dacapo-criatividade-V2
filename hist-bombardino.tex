\subsection*{Bombardino}

A bombardino faz parte da família dos instrumentos de metal
e seu som é produzido através da vibração dos lábios no bocal do
instrumento. Sua origem se encontra na Roma antiga, onde instrumentos
feitos de bronze e metal, chamados ``tubas'', eram usado em funções
militares e cerimoniais. O bombardino, também chamado de euphonium ou
tuba tenor em alguns lugares, apareceu primeiramente na Alemanha na
década de 1830. Um outro instrumento similar ao bombardino é o
barítono, que também foi inventado na década de 1830. O bombardino é a
última versão do saxhorn barítono criado pelo belga Adolphe Sax, o
inventor do saxofone. Seus tubos são mais largos e cônicos que os do
barítono, o qual possui tubos mais cilíndricos. Consequentemente, sua
sonoridade é mais escura que a do barítono.