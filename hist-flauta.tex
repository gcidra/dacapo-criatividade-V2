\subsection*{Flauta}

A flauta faz parte da família dos instrumentos de madeira,
mesmo, ao longo dos anos, tendo sido feita de madeira ou
metal. Sabemos que já existiam flautas nas civilizações antigas. As
flautas antigas eram tocadas apontando para frente, tais como a flauta
doce. A flauta tocada de lado foi chamada de flauta transversal até o
meio do século XIX, depois passou a ser chamada apenas de flauta na
língua inglesa. Em português, é conhecida pelas duas terminologias,
flauta e flauta transversal. A flauta moderna foi projetada na
primeira metade do século XIX por Theobald Boehm. Em relação as
flautas antigas, a moderna possui maior volume sonoro, mais chaves
(possibilitando tocar uma escala cromática completa) e melhor
afinação. Sua família inclui a flauta comum (em dó), o flautim, a
flauta contralto e a baixo. Atualmente, ela faz parte dos mais
variados grupos instrumentais da música erudita e popular (orquestra
sinfônica, banda e grupos de choro, jazz e música folclórica).