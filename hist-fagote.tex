\subsection*{Fagote}

O fagote pertence à família dos instrumentos de madeira e seu
som é produzido através da vibração de uma palheta dupla (duas
palhetas sobrepostas). O ancestral mais velho do fagote, segundo
alguns estudiosos, é chamado dulcian, instrumento de palheta dupla e
feito de uma peça só que tocava a linha do baixo nas músicas do século
XVI. O primeiro fagote feito em mais de uma peça apareceu na França no
século XVII. Carl Almenräder (1786-1843) é considerado como o maior
colaborador do desenvolvimento do fagote moderno. Ele melhorou a
sonoridade e capacidade das notas do instrumento. Em 1831, ele e
A. J. Heckel fundaram uma fábrica que produziu o fagote moderno,
sistema alemão. A sua família inclui o fagote (convencional) e o
contra-fagote, porém no passado ela possuía mais três instrumentos.