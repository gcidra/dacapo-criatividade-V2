\subsection*{Trompete}

O trompete faz parte da família dos instrumentos de metal e
seu som é produzido através da vibração dos lábios no bocal do
instrumento. É o instrumento mais agudo dessa família. Sua origem pode
ser traçada ao antigo Egito, África e Grécia. Antigamente, era
construído de madeira, bronze ou prata e não possuía válvulas
(conhecido como “trompete natural”). Na Idade Média (500-1430), ele
era tocado apenas nas notas graves. Durante a Renascença (1430-1600),
ele foi usado em várias funções cerimoniais. Os trompetistas passaram
a dominar o registro agudo do instrumento aos poucos, especialmente no
período Barroco (1600-1750). Foi em 1814 que Heinrich Stölzel
apresentou o primeiro trompete com válvula. As válvulas permitiram o
instrumento tocar escala cromática. Sua família inclui, entre outros,
os trompetes pícolos em mi bemol e ré, o trompete em dó, o cornet e o
flugelhorn; os tubos dos dois últimos são mais largo e cônico,
consequentemente, possuem uma sonoridade mais escura. Atualmente, ele
faz parte dos mais variados grupos instrumentais da música erudita e
popular (orquestra sinfônica, banda e grupos de choro, jazz e música
folclórica).