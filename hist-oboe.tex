\subsection*{Oboé}

O oboé faz parte da família dos instrumentos de madeira e seu som
é produzido através da vibração de uma palheta dupla (duas palhetas
sobrepostas). Sua origem está relacionada ao shawm do século XIII,
instrumentos de palheta dupla que foram muito usado na música da Idade
Média (500-1430). A palavra oboé é, na verdade, oriunda da palavra
francesa hautbois, a qual designa um instrumento agudo de madeira da
família do shawm. Sua invenção se deu em 1660 e é creditada ao francês
Jean Hoteterre. Hoje, a maioria dos oboés é construída segundo o
sistema de chaves inventado por Theobald Boehm para flauta na primeira
metade do século XIX, prática que se iniciou no século passado. Sua
família inclui o oboé, o oboé d’Amore em lá e o corne inglês em fá.