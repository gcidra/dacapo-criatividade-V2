\subsection*{Saxhorn em mi bemol}

O saxhorn em mi bemol também é chamado de saxgênis
ou saxhorn alto em diferentes partes do Brasil. Ele faz parte da
família dos instrumentos de metal e seu som é produzido através da
vibração dos lábios no bocal do instrumento. Sua origem se encontra na
Roma antiga, onde instrumentos feitos de bronze e metal,
chamados “tubas”, eram usado em funções militares e cerimoniais. Seu
ancestral direto é o bombardino, também chamado de barítono e
conhecido como tuba tenor em alguns lugares, que apareceu
primeiramente na Alemanha na década de 1830. Ele foi inventado por
Adolphe Sax, o inventor do saxofone, que construiu uma família de
saxhornes entre 1843-45, o saxhorn alto, o tenor e o barítono.