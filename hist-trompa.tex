\subsection*{Trompa}

A trompa faz parte da família dos instrumentos de metal e seu
som é produzido através da vibração dos lábios no bocal do
instrumento. A trompa moderna provém da trompa de caça do século
XVI. Esta não possuía válvulas. Ela era conhecida como trompa lisa e
produzia diferentes grupos de notas usando várias partes (“voltas”)
de afinação. Os trompistas tocavam com diversas voltas de afinação ao
seu lado para obter as notas corretas. Ela foi introduzida na França
em 1660, porém foram os fabricantes alemães que aperfeiçoaram a trompa
atual. Em 1818, Stölzel e Bluhmel acrescentaram válvulas a trompa lisa
e eliminaram a necessidade de usar as diversas voltas de afinação. Já
o rotor, que hoje se encontra na maioria desses instrumentos, passaram
a ser empregados em 1853. A trompa simples, em fá, contem três
válvulas e a dupla, fá/si bemol, quatro.