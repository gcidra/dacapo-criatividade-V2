\subsection*{Saxofone}

O saxofone faz parte da família dos instrumentos de madeira e seu som
é produzido através da vibração de uma palheta simples. Foi inventado
pelo belga Adolphe Sax na década de 1840 e divulgado mais intensamente
na França, durante este período. Sua família inclui o saxofone soprano
em mi bemol, o saxofone alto em mi bemol, o saxofone tenor em si
bemol, o saxofone barítono em mi bemol e o saxofone baixo em si
bemol. O dedilhado destes instrumentos são semelhantes, facilitando
sua execução. Ele foi bem explorado no jazz e utilizado por vários
compositores famosos no repertório orquestral.